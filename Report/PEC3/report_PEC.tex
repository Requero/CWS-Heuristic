% !TEX TS-program = pdflatex
% !TEX encoding = UTF-8 Unicode

% This is a simple template for a LaTeX document using the "article" class.
% See "book", "report", "letter" for other types of document.

\documentclass[11pt]{article} % use larger type; default would be 10pt

\usepackage[utf8]{inputenc} % set input encoding (not needed with XeLaTeX)
\usepackage{caption}
\usepackage{subcaption}
\usepackage{lscape} 
\usepackage{rotating}
\usepackage{longtable}
\usepackage{graphicx}
\usepackage{float}
%%% Examples of Article customizations
% These packages are optional, depending whether you want the features they provide.
% See the LaTeX Companion or other references for full information.

%%% PAGE DIMENSIONS
\usepackage{geometry} % to change the page dimensions
\geometry{a4paper} % or letterpaper (US) or a5paper or....
% \geometry{margin=2in} % for example, change the margins to 2 inches all round
% \geometry{landscape} % set up the page for landscape
%   read geometry.pdf for detailed page layout information

\usepackage{graphicx} % support the \includegraphics command and options

% \usepackage[parfill]{parskip} % Activate to begin paragraphs with an empty line rather than an indent

%%% PACKAGES
\usepackage{booktabs} % for much better looking tables
\usepackage{array} % for better arrays (eg matrices) in maths
\usepackage{paralist} % very flexible & customisable lists (eg. enumerate/itemize, etc.)
\usepackage{verbatim} % adds environment for commenting out blocks of text & for better verbatim
% These packages are all incorporated in the memoir class to one degree or another...
\setlength{\textwidth}{150mm}
\setlength{\textheight}{250mm}
\setlength{\oddsidemargin}{6mm}
\setlength{\evensidemargin}{28mm}
\setlength{\topmargin}{-15mm}
%%% HEADERS & FOOTERS
\usepackage{fancyhdr} % This should be set AFTER setting up the page geometry
\pagestyle{fancy} % options: empty , plain , fancy
\renewcommand{\headrulewidth}{0pt} % customise the layout...
\lhead{Simheuristics}\chead{}\rhead{PEC3}
\lfoot{}\cfoot{\thepage}\rfoot{}

%%% SECTION TITLE APPEARANCE
\usepackage{sectsty}
\allsectionsfont{\sffamily\mdseries\upshape} % (See the fntguide.pdf for font help)
% (This matches ConTeXt defaults)

%%% ToC (table of contents) APPEARANCE
\usepackage[nottoc,notlof,notlot]{tocbibind} % Put the bibliography in the ToC
\usepackage[titles,subfigure]{tocloft} % Alter the style of the Table of Contents
\renewcommand{\cftsecfont}{\rmfamily\mdseries\upshape}
\renewcommand{\cftsecpagefont}{\rmfamily\mdseries\upshape} % No bold!

%%% END Article customizations

%%% The "real" document content comes below...

\title{Simheuristics}
\author{Alvarez Estarlich, Mauro \\ Requero Martín, David \\ Serrano Gómez, Enrique}
%\date{} % Activate to display a given date or no date (if empty),
         % otherwise the current date is printed 

\begin{document}
\maketitle

\section{Resumen y Comentario}

\subsection{A simheuristic algorithm for solving the permutation flow shop problem with stochastic processing times (Juan, A. et. al. (2014))}

El Flow Shop Scheduling Problem (FSP) es un problema que busca, en la mayoría de planteamientos, reducir el tiempo de producción de un conjunto de máquinas que tienen asociados N trabajos a realizar. Una variación del FSP es el Permutation Flow Shop Scheduling Problem (PFSP) que busca encontrar, realizando permutaciones en el orden de los trabajos, reducir el tiempo requerido para completar las tareas. Por último, el objeto de trabajo del articulo se centra en la generalización del PFSP, el PFSP with Stochastic Times (PFSPST) que añade el uso de valores aleatorios a la hora de determinar el tiempo de procesado de cada trabajo.\\[0.2cm]
La literatura relacionada con la variante estocástica del PFSP no es muy extensa, pues la mayor parte de artículos se han centrado en la variante determinista del problema. Por ello, algoritmo propuesto en el articulo para resolver este problema se basa en aprovechar los avances hechos en el estudio de la variante determinista para poder aplicarlos a la variante estocástica.\\[0.2cm]
Concretamente, el algoritmo propuesto consta de las siguientes etapas:
\begin{enumerate}
\item Convertir el problema en la variante determinista (PFSP, sin tiempos aleatorios).
\item Obtener soluciones de calidad, utilizando algoritmos eficientes basados en el PFSP (e.g. Iterated Local Search).
\item Ejecutar una simulación con las soluciones obtenidas, para obtener una muestra de los “tiempos para completar los trabajos”.
\item Repetir los puntos 2-3 hasta que las condiciones de parada del algoritmo se cumplan (ya sea por tiempo o por calidad de la solución). 
\end{enumerate}

\textbf{Valoración}

Desde mi punto de vista lo más destacable seria:
\begin{itemize}
\item En primer lugar, la simplificación del problema de la variante estocástica en la variante determinista resulta una estrategia muy beneficiosa ya que como se afirma en el artículo, la solución del modelo determinista necesariamente estará dentro del espacio de soluciones del modelo estocástico. Además, trabajando con la variante determinista se puede sacar provecho de los mejores resultados obtenidos en otros estudios. 
\item En segundo lugar, el uso de simulaciones para obtener los diferentes “tiempos de finalización de tareas” aporta la ventaja de no depender del uso de distribuciones estándar a la hora de determinar los tiempos de ejecución de las diferentes tareas, pues estas pueden no seguir dichas distribuciones por lo que se pueden ir adaptando a medida que el algoritmo se ejecuta.
\end{itemize}

\subsection{A simheuristic algorithm for the Single-Period Stochastic Inventory-Routing Problem with stock-out(Juan, A. et. al. (2014))}


\textbf{Valoración}\\[0.2cm]


\subsection{A review of simheuristics: Extending metaheuristics to deal with stochastic combinatorial optimization problems (Juan, A. et. al. (2015))}

\textbf{Simheuristics como metodología de optimización de simulación}\\[0.2cm]
Hay 2 enfoques diferentes:

\begin{itemize}
\item \underline{Simulation-Optimization (SO)}:\\[0.2cm]
El proceso de optimización utiliza datos de salida del modelo de simulación, que evalúa el rendimiento de la solución dada. Esa solución consiste en una serie de decisiones, que son los datos de entrada del modelo de simulación. Basado en esta y pasadas evaluación, el proceso de optimización decide sobre un nuevo conjunto de datos de entrada. Este modelo de simulación actúa como una Evaluation Function (EF)  del proceso de optimización.
Alternativamente, varias soluciones pueden ser simuladas para construir un modelo alternativo (SCM) que puede resolverse utilizando técnicas de optimización clásicas en vez de simulación. La solución a este metamodelo es considerada como una solución aproximada al problema original.
\item \underline{Hybrid Simulation-Analytic (HSA)}:\\[0.2cm]
Un modelo HSA es cualquier programa estocástico con escenarios muestreados. La simulación HSA es para potenciar el modelo analítico (AME) o para generar parte de la solución (SG).
En AME, la simulación se usa para refinar los parámetros de un modelo analítico de un problema específico. Este método necesita menor número de iteraciones en la simulación que EF.
\end{itemize}

\textbf{Lógica detras de Simheuristics}\\[0.1cm]

Este algoritmo sirve para afrontar eficientemente problemas de optimización combinatoria (COPs) que contienen componentes estocásticos. \\[0.1cm]

Este enfoque asume que, en escenarios con incerteza moderada, soluciones de alta calidad para la versión determinista del COP, probablemente serán soluciones de alta calidad para su correspondiente versión estocástica.\\[0.1cm]

Dada una instancia COP estocástica, su contraparte determinista es considerada. Después un algoritmo basado en metaheurística es ejecutado para realizar una búsqueda eficiente dentro del espacio de soluciones asociado al COP. Este proceso tiene como objetivo encontrar un conjunto de soluciones factibles de alta calidad para el COP determinista.\\[0.1cm]

Solo soluciones ‘prometedoras’ son enviadas al componente de simulación, lo que permite controlar el esfuerzo computacional empleado por la simulación durante el proceso de búsqueda.\\[0.1cm]

Valores estimados por la simulación pueden ser usados para mantener una lista de las mejores soluciones para el problema estocástico. Esto puede retroalimentar el metaheurístico para que intensifique la exploración de áreas del espacio de soluciones prometedoras.\\

\textbf{Conclusión}\\[0.2cm]
En vez de un enfoque de caja negra, donde las evaluaciones son realizadas solo por la simulación, los Simheuristics integran estrechamente la optimización y la simulación mediante la incorporación de información específica del problema.\\[0.2cm]


\textbf{Valoración}\\[0.2cm]

El artículo tiene una estructura simple y acertada porque te hace una introducción a lo que son los simheuristics y además te aporta muchos ejemplos de aplicaciones de metaheurística y simulación en diversos ámbitos. Una vez ya has visto diferentes aplicaciones te explica la lógica detrás de una combinación más estrecha de la simulación y la metaheurística, y el potencial que tiene este enfoque en diferentes ámbitos que han sido tratados anteriormente en el artículo.\\[0.2cm]
El artículo no te abruma con muchísima información diferente, todo gravita a la simulación y la metaheurística lo que ayuda a no perderse y las explicaciones junto con los ejemplos ayudan a que el tema puede entenderse más fácilmente.\\[0.2cm]
Tiene una conclusión que ayuda a retener la idea del artículo y esta referenciado correctamente.


\subsection{Learnheuristics: hybridizing metaheuristics with machine learning for optimization with dynamic inputs (Calvet, L. et. al. (2017))}

Este artículo revisa la literatura existente relacionada con la combinación de metaheurísticos y métodos de Machine Learning (ML) y luego presenta el concepto de Learnheuristics.\\[0.2cm]

\textbf{Uso de ML para mejorar Metaheurísticos:}

\begin{itemize}
\item \textbf{Hibridaciones específicamente localizadas:} Donde ML se aplica en un proceso especifico.\\[0.2cm]
La puesta a punto de parámetros metaheurísticos es conocida por tener un efecto significativo en el rendimiento de algoritmos. Hay 3 enfoques/estrategias:
\begin{itemize}
\item Estrategias de control de parámetros.
\item Estrategias de puesta a punto de parámetros.
\item Estrategias de puesta a punto de parámetros en instancias específicas.
\end{itemize}
Respecto a la gestión de población, los autores intentan extraer información de soluciones que ya han visitado y la utilizan para crear nuevas soluciones para explorar espacios de búsqueda más prometedores.\\

\item \textbf{Hibridaciones globales:} ML tiene un efecto mayor en el diseño metaheurístico.\\[0.2cm]
Algorithm selection problem (ASP)  tiene como objetivo predecir que algoritmo, dentro de un conjunto de algoritmos, va a tener un mejor rendimiento. Una red neural implementando una selección de parámetros suavizada es entrenada para predecir el mejor algoritmo.\\[0.2cm]
Hyper heuristics pueden ser descritos como métodos de búsqueda o mecanismos de aprendizaje para seleccionar o generar heurísticos para soluciones problemas computacionales de búsqueda. 
\item \textbf{Utilizando metaheurísticos para mejorar ML}\\[0.2cm]
Classification, Metaheurísticos han sido principalmente aplicados para la selección de atributos, extracción de atributos y puesta a punto de parámetros.\\[0.2cm]
Regression, Aplicados al uso de ML relacionado con el entrenamiento de modelos de regresión compleja. \\[0.2cm]
Clustering, Algoritmos de evolución diferencial son aplicados a este tipo de problemas. 
\item \textbf{Marco de trabajo LearnHeuristic}\\[0.2cm]
Este tiene como objetivo resolver problemas de optimización combinatoria (COPs) en los cuales los datos de entrada del modelo no son fijos de antemano. En lugar de eso, estos datos pueden variar de manera predecible de acuerdo con el estado de la solución parcialmente construida en cada iteración del heurístico de construcción. \\[0.2cm]
El objetivo de este tipo de problemas es minimizar las funciones de coste en relación a ciertos límites. La característica novedosa es que los datos de entrada de la función objetivo y/o los límites pueden depender de la estructura de la solución, lo que hace que estos datos sean dinámicos a medida que la solución parcial evoluciona.
\end{itemize}

\textbf{Valoración}\\[0.2cm]

Del artículo me ha parecido que el orden en el que aborda cada apartado es correcto. Va a hablar de metaheurística y Machine Learning y primero explica que es cada cosa, luego hace una revisión de la combinación de estas y luego explica el trabajo que hacen ellos con LearnHeuristics.\\[0.2cm]
Además del orden, las referencias utilizadas son muy diversas y da la oportunidad de profundizar en todo lo que menciona.\\[0.2cm]
Como punto negativo hay que mencionar que el articulo puede ser un poco denso y que ciertas partes, como la revisión de artículos, podrían haberse esquematizado mejor para que la lectura fuera más clara y amena.

\clearpage

{\fontsize{50}{60}\selectfont Diseño y desarrollo de un algoritmo de XXXXXX aplicada al VRP}

\renewcommand{\labelenumi}{\arabic{enumi}}
 \begin{enumerate}
   \item) \textbf{Introducción}\\[0.2cm]

   \item) \textbf{Revisión de la literatura}\\[0.2cm]

   \item) \textbf{Algoritmo desarrollado}\\[0.2cm]

   \item) \textbf{Experimento computacional y análisis de resultados}\\[0.2cm]

   \item) \textbf{Conclusiones y trabajo futuro}\\[0.2cm]


 \end{enumerate}

\end{document}
